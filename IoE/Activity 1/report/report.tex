\documentclass[journal]{IEEEtran}
\usepackage{cite}
\usepackage{amsmath,amssymb,amsfonts}
\usepackage{algorithmic}
\usepackage{graphicx}
\usepackage{textcomp}
\usepackage{xcolor}


\begin{document}

\title{Activity 1: Stochastic Simulations of Molecular Communications }

\author{Blind grading number 2504F}


\maketitle

\begin{abstract}
This document is a model and instructions for \LaTeX.
You may use it as a template for your own writing. Here, you write the abstract for your paper.
\end{abstract}

\begin{IEEEkeywords}
IEEE, IEEEtran, journal, \LaTeX, paper, template.
\end{IEEEkeywords}

\section{Introduction}
% The very first letter is a 2 line initial drop letter followed by the rest of the first word in caps.
\IEEEPARstart{T}{his} document is a model and instructions for \LaTeX.
You can use it as a template for your own writing. Use the template as is, or adjust it according to your own needs.

\section{Main Content}
Detail your main content here. Break it into sections as needed.

\subsection{Subsection Example}
Provide details specific to a subsection here.

\subsubsection{Subsubsection Example}
Further details can be broken down into subsubsections.

\section{Conclusion}
The conclusion goes here.

% Note that the IEEE does not put floats in the very first column - or typically anywhere on the first page for that matter. Also, in-text middle ("here") positioning is typically not used, but it is allowed and encouraged for Computer Society conferences (but not Computer Society journals). Most IEEE journals/conferences use top floats exclusively. Note that, LaTeX2e, unlike IEEE journals/conferences, places footnotes above bottom floats. This can be corrected via the \fnbelowfloat command of the stfloats package.

\section*{Acknowledgment}

The authors would like to thank...

% References section
\bibliographystyle{IEEEtran}
% argument is your BibTeX string definitions and bibliography database(s)
\bibliography{IEEEabrv,mybibfile}

% that's all folks
\end{document}
