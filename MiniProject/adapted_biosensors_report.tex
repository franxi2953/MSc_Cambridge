\documentclass[12pt]{article}
\usepackage[utf8]{inputenc}
\usepackage{graphicx}
\usepackage{caption}

% Set the margins
\usepackage[a4paper, total={6in, 8in}]{geometry}


\newcommand\wordcount{
   \immediate\write18{wordcount.bat \jobname.tex}
   \input{\jobname.wc}
}

\begin{document}

% Title Page
\begin{titlepage}
    \centering
    \vspace*{60px}
    \huge{Title of the Project}\\
    \vspace{10px}
    \large{Your Name}\\
    \large{Names of Academic/Company Supervisors}\\
    \vfill
    \today
    \vfill
\end{titlepage}

% Abstract
\newpage
\begin{abstract}
\noindent
Your abstract here. (150-250 words)
\end{abstract}

% Main Body
\newpage
\section*{Introduction}
\subsection*{Papers to read}
\subsubsection*{Digital Microfluidics}

Paper describing the OpenDrop

moving proteins

the next is bioactivating the surface

but wait... it is covered by a fluoropolimer!
strategies
 - can we bind anything to the fluoropolimer?
  - can we scratch it and bind it to the thing below


\subsubsection*{Protein lithography}
¿Could it also be implemented lipid photolithography?
How i imagine the process?
 - First doing lithography of proteins. 
      - The lithography can help us patterning little island that does not overcrowd the glass, so there's later space for joining enough lipids and making everything hidrophobic.
 - The covering the rest of the space with lipids

 First I would need to characterize if the lipids (and which lipid) makes the glass hidrophobic enough.

 The above mentioned can not be done! For binding lipids we need silane activation (Or maybe plasma?) which would destroy proteins already present.
 Maybe we can first bind the lipids (Or maybe actually the fluoropolimer!), then remove them with a laser in precise locations, then do silane activation 
 (Or whatever it's needed for binding proteins) and then bind the proteins directly to the glass? In this way the protein would be directly in contact with the glass.


 Si, parece que la forma más sencilla va a ser:
 1) Cubrir los cristales y los chips con el fluoropolimero
 2) hacer pequeños agugeros con el laser
 3) poner proteina en estos pequeños agugeros
 
 Una alternativa potencial es hacer fotolitografía en el cristal con un polimero positivo para que el fluoropolimero no se deposite?
 Y despues remover la mascara de alguna forma y que con ella se lleve el teflon que hay encima.... mmm va a ser un problema porque el teflon ya lo esta recubriendo todo...


 \subsubsection*{capacitive sensors}

% Figures
% \begin{figure}[ht]
% \centering
% \includegraphics[width=0.5\textwidth]{path/to/your/image.png}
% \caption{Figure caption (adapted from [source]).}
% \label{fig:my_label}
% \end{figure}

\wordcount

% References
\newpage
\bibliographystyle{IEEEtran}
\bibliography{report2_biosensors}

\end{document}


\definecolor{myblue}{HTML}{717AFD}
\definecolor{myred}{HTML}{F0564A}
\definecolor{mygrey}{HTML}{464546}

\setlength{\parskip}{0.8em} % Adjust space between paragraphs

