\documentclass[10pt, twocolumn]{article}
\usepackage[utf8]{inputenc}
\usepackage{graphicx}
\usepackage{caption}

% Set the margins
\usepackage[a4paper, total={6in, 8in}]{geometry}


\newcommand\wordcount{
   \immediate\write18{wordcount.bat \jobname.tex}
   \input{\jobname.wc}
}

\begin{document}

% Title Page
\begin{titlepage}
    \centering
    \vspace*{60px}
    \huge{Title of the Project}\\
    \vspace{10px}
    \large{Francisco Javier Quero}\\
    \large{Names of Academic/Company Supervisors}\\
    \vfill
    \today
    \vfill
\end{titlepage}

% Abstract
\clearpage
\onecolumn

\begin{abstract}
\noindent
Your abstract here. (150-250 words)
\end{abstract}

\clearpage
\twocolumn

% Main Body
\newpage
\section*{Introduction}

Digital microfluidics (DMF) is a field dedicated to automating laboratory procedures by harnessing a variety of technologies to facilitate an interface between programmable digital software and the controlled movement of fluids at the microfluidic scale.  In this project, we will specifically focus on a branch of DMF known as electrowetting-on-dielectric (EWOD) which utilizes the positioning of electrical charges beneath a thin, highly hydrophobic insulating layer to locally reduce the repulsion forces between the layer and the polar fluid droplets, thus facilitating their controlled movement \cite{beniContinuousElectrowettingEffect1982}. In the past ten years, substantial investments have been made in the field by both public organizations and private corporations. A notable example is Oxford Nanopore, which has redirected much of its funding to their digital microfluidics system, VolTRAX, aimed at providing a straightforward solution for automated sequencing library preparation \cite{VolTRAX2018,OxfordNanoporeAnnounces}. Such financial backing has precipitated swift technological progress, manifesting in both the simplification and cost-reduction of hardware \cite{zhang2DLargescaleEWOD2020} as well as the advent of new technologies that enable greater miniaturization and an increased scaling of electrode numbers\cite{qinSolutionMassProduction2021}. Currently, the principal challenge for EWOD is the integration of sensors into the chips, particularly chemical and biosensors, since most relevant sensing technologies are dependent on costly and cumbersome optical microscopy apparatus.

In this mini-project, we aim to repurpose an already developed method of protein patterning on surfaces to bioactivate the crystals covering the microdroplets\cite{straleMultiproteinPrintingLightInduced2016}. The main goal is to establish a straightforward protocol for easily attaching biomolecules to the surface in contact with the fluid, driving specific molecular interactions between the electrowetting device's surface and the fluid's molecules. Lastly, we plan to investigate how this binding mechanism influences the feedback system of the platform, which measures changes in capacitance related to the bound state of the molecules [SOLVE]. This project will explore potential alterations in the capacitance signal based on the binding of droplet molecules to the patterned biomolecules on the glass \cite{wangUltraSensitiveCapacitiveMicrowire2019}.

The practical benefits of a platform capable of biologically interacting with the fluids it manipulates are evident. Whether as an actuator, selectively binding (and quelating) proteins, cells, or other compounds from a solution, or as a sensor, measuring real-time factors such as cell types proportions, DNA/RNA elements, or molecular concentrations in cell-containing or cell-free mixtures. In this way, we aim to bring EWOD-based digital microfluidics beyond mere fluid movement automation, becoming a comprehensive platform that measures and influences the biochemical layer of these reactions.


\subsection*{The paltform: OpenDrop v4}
- Tradicionalmente la microfluidica digital a estado sujeta bien a sistemas DIY creados de forma casera en laboratorios, o bien sistemas altamente caros diseñados para un trabajo experimental altamente especifico de laboratorios highly funded / equipped. 
- OpenDrop viene a solucionar este problema, usando metodos de fabricación estandar de PCB (facilmente escalable) y hardware open source (Que facilita la adaptación de partes específicas del hardware), opendrop se presenta como una plataforma sencilla y barata de replicar y adaptar con unos niveles de precisión comparables a otras plataformas descritas en la literatura.
- Además la versión 4 de OpenDrop (cita) es un sistema modular basado en "Cartridges". El investigador que la use no tiene que diseñar todo el dispositivo, si no solo el cartridge, que incluye el patron de los electrodos y toda la sensorica en torno a ellos, que se conecta a la main board a traves de conectores estandares que permiten drive el sistema. Esta modularidad es clave porque nos permite eventualmente diseñar nuestros propios cartridges sin tener que reinventar el sistema de control principal.

\subsection*{Patterning proteins}
El principal problema; el ITO glass, el substrato donde se unirían las proteinas, debe de estar (al menos mayoritariamente) cubierto de una capa hidrofóbica, para permitir que la gota se mueva gracias a las interacciones hidrofilicas entre el fluido y la repulsion de la capa de fluoropolimero, excepto en las zonas que se encuentran sobre los electrodos (temporalmente hidrofilicas). Por otro lado, como las proteinas se unen al cristal y no al fluoropolimero, estas han de exponerse al fluido, por lo tanto la capa hidrofobica tiene que ser removida localmente para permitir exponer estas proteinas, pero de una forma suficientemente sutil para que la interaccion gota-cristal no apantalle la potencial repulsion de la capa hidrofobica una vez el electrodo se apague.

\subsubsection*{Exposing the ITO glass over the hydrophobic cover layer}
Dos métodos potenciales; (1) Cubrir todo el cristal y desgastar las zonas que queremos con un laser (cita) o (2) usar un fluoropolimero que polimerice con UV, exponiendo solo parte del cristal al proceso de solidificación.
\subsubsection*{Proteing binging and quality control}
\subsection*{Signal generation}

Paper describing the OpenDrop

moving proteins

\cite{strale_multiprotein_2016}

the next is bioactivating the surface

but wait... it is covered by a fluoropolimer!
strategies
 - can we bind anything to the fluoropolimer?
  - can we scratch it and bind it to the thing below


\subsubsection*{Protein lithography}
¿Could it also be implemented lipid photolithography?
How i imagine the process?
 - First doing lithography of proteins. 
      - The lithography can help us patterning little island that does not overcrowd the glass, so there's later space for joining enough lipids and making everything hidrophobic.
 - The covering the rest of the space with lipids

 First I would need to characterize if the lipids (and which lipid) makes the glass hidrophobic enough.

 The above mentioned can not be done! For binding lipids we need silane activation (Or maybe plasma?) which would destroy proteins already present.
 Maybe we can first bind the lipids (Or maybe actually the fluoropolimer!), then remove them with a laser in precise locations, then do silane activation 
 (Or whatever it's needed for binding proteins) and then bind the proteins directly to the glass? In this way the protein would be directly in contact with the glass.


 Si, parece que la forma más sencilla va a ser:
 1) Cubrir los cristales y los chips con el fluoropolimero
 2) hacer pequeños agugeros con el laser
 3) poner proteina en estos pequeños agugeros
 
 Una alternativa potencial es hacer fotolitografía en el cristal con un polimero positivo para que el fluoropolimero no se deposite?
 Y despues remover la mascara de alguna forma y que con ella se lleve el teflon que hay encima.... mmm va a ser un problema porque el teflon ya lo esta recubriendo todo...


 \subsubsection*{capacitive sensors}

% Figures
% \begin{figure}[ht]
% \centering
% \includegraphics[width=0.5\textwidth]{path/to/your/image.png}
% \caption{Figure caption (adapted from [source]).}
% \label{fig:my_label}
% \end{figure}

\wordcount

% References
\newpage
\bibliographystyle{IEEEtran}
\bibliography{MiniProjectCDT.bib}

\end{document}


\definecolor{myblue}{HTML}{717AFD}
\definecolor{myred}{HTML}{F0564A}
\definecolor{mygrey}{HTML}{464546}

\setlength{\parskip}{0.8em} % Adjust space between paragraphs

